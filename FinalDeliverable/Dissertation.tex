\documentclass[11pt]{article}
\usepackage[table]{xcolor}
\usepackage[margin=1in]{geometry}
\usepackage{amsmath}
\usepackage{graphicx}

\usepackage{setspace}
\doublespacing

\usepackage{amssymb}
\usepackage{epstopdf}
\usepackage{inputenc}

\usepackage{dashrule}
\usepackage{float}
\usepackage{hyperref}
\usepackage{url}
\usepackage{mwe}
\usepackage{caption}
\usepackage{booktabs}
\usepackage{multirow}
\usepackage{enumitem}

\usepackage[backend=biber,style=ieee]{biblatex}
\usepackage[toc]{appendix}
\usepackage[acronym]{glossaries}

\addbibresource{Dissertation.bib}

\hypersetup{ linktoc=all}
\graphicspath{ {./images/}{../images/}{../../images} }

%\makenoidxglossaries

\newacronym{ai}{AI}{artifical intelligence}
\newacronym{fc}{FC}{fully connected}
\newacronym{dnn}{DNN}{deep neural network}
\newacronym{cnn}{CNN}{convolutional neural network}
\newacronym{rnn}{RNN}{recurrent neural network}
\newacronym{lstm}{LSTM}{long term short term memory network}
\newacronym{nlp}{NLP}{natural language processing}

\newacronym{ncs}{NCS}{neural compute stick}
\newacronym{tpu}{TPU}{tensor processing unit}
\newacronym{vpu}{VPU}{video processing unit}
\newacronym{gpu}{GPU}{graphics processing unit}
\newacronym{apu}{APU}{associative processing unit}
\newacronym{fpga}{FPGA}{field programmable gate array}
\newacronym{asic}{ASIC}{application specific integrated circuit}

\newacronym{blas}{BLAS}{basic linear algebra subprograms}
\newacronym{csc}{CSC}{compressed sparse column}
\newacronym{tops}{TOPS}{trillion operations per second}
\newacronym{soc}{SoC}{system on a chip}


\usepackage{subfiles}

\begin{document}
\title{%
	\bf Inference at the edge: tuning compression parameters for performance\\ 
	\large Deliverable 1: Final year Dissertation \\
	Bsc Computer Science: Artificial Intelligence}

\author{
	Sam Fay-Hunt | \texttt{sf52@hw.ac.uk}\\
	Supervisor: Rob Stewart | \texttt{R.Stewart@hw.ac.uk}
}

\maketitle
\thispagestyle{empty}
\pagebreak

\textbf{DECLARATION}\\
I, Sam Fay-Hunt confirm that this work submitted for assessment is my own and is expressed in
my own words. Any uses made within it of the works of other authors in any form (e.g., ideas,
equations, figures, text, tables, programs) are properly acknowledged at any point of their
use. A list of the references employed is included.\\
Signed: .....Sam Fay-Hunt...........\\
Date: .....10/12/2020...............
\thispagestyle{empty}
\pagebreak

\textbf{Abstract:} 
\emph{Abstract here}



\thispagestyle{empty}
\pagebreak

\tableofcontents
\thispagestyle{empty}
\pagebreak

%\printnoidxglossary[type=acronym, nonumberlist]
%\thispagestyle{empty}

\newpage
\setcounter{page}{1}

\section{Introduction}
%\subfile{Sections/introduction}
\emph{
\begin{itemize}
	\item Introduce terminology  - Inference, neural network model, pruning, layers, channels, filters
	\item Introduce models to be used - high level conceptual representation of the models
	\item Introduce hypothesis
	\item Describe research aims
	\item Define project objectives
	\item Describe how this work contributes to further research
\end{itemize}
}


\pagebreak
\section{Background}
\emph{
\begin{itemize}
	\item Adapt from D1
	\item rewrite with more of a focus on the concrete channel and pruning methodology used
	\item Would be good to include wandb bayse hyperparam optimisation details
\end{itemize}
}
%This Section will be split into 4 subsections:\\
%Section~\ref{subsec:deepLearning} - \textbf{Deep Learning}: An overview of the basic components of a deep neural network and the \acrshort{cnn} model.\\
%Section~\ref{subsec:compressionTypes} - \textbf{Neural Network Compression}: Discusses neural network compression techniques and on how they change the underlying representations of DNNs.\\
%Section~\ref{subsec:AIaccelerators} - \textbf{AI accelerators} Covers a few popular AI accelerators architectures, their strengths, weaknesses and specialisms.\\
%Section~\ref{subsec:hardwareArch} - \textbf{Memory factors for Deep Neural Networks}: Describes the how DNNs interact with memory, and discusses some of the implications of this.

%\subsection{Deep Neural Networks}\label{subsec:deepLearning}
%\subfile{Sections/Background/DeepNeuralNetworks}

%\newpage
%\subsection{Neural Network Compression}\label{subsec:compressionTypes}
%\subfile{Sections/Background/Compression}

%\newpage
%\subsection{AI accelerators}\label{subsec:AIaccelerators}
%\subfile{Sections/Background/AIaccelerators}

%\newpage
%\subsection{Memory factors for Deep Neural Networks}\label{subsec:hardwareArch}
%\subfile{Sections/Background/HardwareMemArch}

%\newpage



\section{Methodology}
\subfile{Sections/Methodology}


\section{Evaluation}
\subsection{Evaluation of experimental design}
\emph{
\begin{itemize}
	\item Duration of training
	\item volume of data gathered
	\item (im)practicalities - power consumption? 
	\item limitations - single optimisation metric
	\item Criticism of methodology
\end{itemize}
}

\subsection{Evaluation of results}
\emph{
\begin{itemize}
	\item Summary of results per model/dataset
	\item Deep dive into results, detailed visualisations of accuracy \& latency tradeoffs (maybe example with poor quality sensitivity analysis vs higher quality layer selection)
	\item 
\end{itemize}
}

\section{Conclusion}
\subsection{Further work}
\emph{
\begin{itemize}
	\item Suggested improvements for methodology
	\item Next steps
\end{itemize}
}
\subsection{Discussion}
\emph{
\begin{itemize}
	\item Discuss results
\end{itemize}
}

\appendix
\section{Back matter}
\subsection{References}
\printbibliography


\end{document}
