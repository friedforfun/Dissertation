\documentclass[../D1.tex]{subfiles}

\begin{document}
This is required for research projects and should be linked
back to the project aim and objectives. It should describe the research methods that
will be employed in the project and the research questions that will be investigated.

\subsection{Research questions}
The following are the research questions this research will seek to cover:

\begin{itemize}
    \item Application of pruning algorithms on neural networks: how does the compression paradigm impact latency?
    \item Can we apply hyperparameter optimisation methods to neural network compression to minimize latency?
    \item 
\end{itemize}


\subsection{Preliminary Evaluation}\label{sec:prelimEval}
To demonstrate the necessity of \hyperref[obj:VerifyComp]{objective O0}, this section presents findings from a series of preliminary benchmarks. 
According to the literature covered in section~\ref{sec:Pruning} course-grained pruning algorithms should provide a demonstrable improvement in latency during inference.
Likewise given the hardware requirements quantisation is an even more consistent in its ability to reduce inference latency (see section~\ref{sec:Quantisation}).

Table~\ref{tab:PrelimResults} presents findings when benchmarking inference of resnet20 with the CIFAR10 dataset on the NCS (table ref), these results show no real change between compression methods, this is an unexpected result.

\begin{table}[H]
    \begin{tabular}{@{}|p{5cm}|p{2cm}|p{2cm}|p{2cm}|p{2cm}|@{}}
    \toprule
    Compression algorithm                     & Top 1 Accuracy & Top 5 Accuracy & Latency (ms) & Throughput (FPS) \\ \midrule
    N/A (baseline)                            & 91.120         & 99.660         & 10.19        & 392.22           \\ \midrule
    AGP filter, fine-grained, and row pruning & 91.110         & 99.700         & 10.15        & 394.14           \\ 
    \midrule
    ssl channels removal                      & 91.610         & 99.780         & 10.17        & 389.17           \\
    \bottomrule
    \end{tabular}
    \caption{Preliminary NCS inference results, Resnet20 trained and tested with the CIFAR10 dataset.}
    \label{tab:PrelimResults}
\end{table}

\emph{Experiment stage 0} in Section~\ref{sec:Experiment0} will investigate this further with the aim of verifying proper application of the compression scheduler to the model, and assessing factors in transferring the model to the NCS.

\subsection{Research methodology}


\subsubsection{Compressing}
Distiller LR scheduling

\subsubsection{Benchmarking}
Openvino benchmarking tool

\subsubsection{Optimisation}
bayesian optimisation \autocite{snoekPracticalBayesianOptimization2012}



\end{document}