\documentclass[11pt]{article}
\usepackage[table]{xcolor}
\usepackage[margin=1in]{geometry}
\usepackage{amsmath}
\usepackage{graphicx}

\usepackage{setspace}
\doublespacing

\usepackage{amssymb}
\usepackage{epstopdf}
\usepackage{inputenc}

\usepackage{dashrule}
\usepackage{float}
\usepackage{hyperref}
\usepackage{url}
\usepackage{mwe}
\usepackage{caption}

\usepackage[backend=biber,style=ieee]{biblatex}
\usepackage[toc]{appendix}
\usepackage[acronym]{glossaries}

\addbibresource{D1.bib}

\hypersetup{ linktoc=all}
\graphicspath{ {./images/}{../images/}{../../images} }

\makenoidxglossaries

\newacronym{ai}{AI}{artifical intelligence}
\newacronym{fc}{FC}{fully connected}
\newacronym{dnn}{DNN}{deep neural network}
\newacronym{cnn}{CNN}{convolutional neural network}
\newacronym{rnn}{RNN}{recurrent neural network}
\newacronym{lstm}{LSTM}{long term short term memory network}
\newacronym{nlp}{NLP}{natural language processing}

\newacronym{ncs}{NCS}{neural compute stick}
\newacronym{tpu}{TPU}{tensor processing unit}
\newacronym{vpu}{VPU}{video processing unit}
\newacronym{gpu}{GPU}{graphics processing unit}
\newacronym{apu}{APU}{associative processing unit}
\newacronym{fpga}{FPGA}{field programmable gate array}
\newacronym{asic}{ASIC}{application specific integrated circuit}

\newacronym{blas}{BLAS}{basic linear algebra subprograms}


\usepackage{subfiles}

\begin{document}
\title{%
	\bf Inference at the edge: the impact of compression on performance\\ 
	\large Deliverable 1: Final year Dissertation \\
	Bsc Computer Science: Artificial Intelligence}

\author{
	Sam Fay-Hunt | \texttt{sf52@hw.ac.uk}\\
	Supervisor: Rob Stewart | \texttt{R.Stewart@hw.ac.uk}
}

\maketitle
\thispagestyle{empty}
\pagebreak

\textbf{DECLARATION}\\
I, Sam Fay-Hunt confirm that this work submitted for assessment is my own and is expressed in
my own words. Any uses made within it of the works of other authors in any form (e.g., ideas,
equations, figures, text, tables, programs) are properly acknowledged at any point of their
use. A list of the references employed is included.\\
Signed: ......................\\
Date: .........................
\thispagestyle{empty}
\pagebreak

\textbf{Abstract:} a short description of the project and the main work to be carried out – probably
between one and two hundred words
\thispagestyle{empty}
\pagebreak

\tableofcontents
\thispagestyle{empty}
\pagebreak


\setcounter{page}{1}

\section{Introduction}
\subfile{Sections/introduction}

\pagebreak
\section{Background}
\emph{Discussing related work found in the technical literature and its relevance to your project.}
This Section will be split into 4 subsections:\\
Section~\ref{subsec:deepLearning} - \textbf{Deep Learning}: An overview of the basic components of a neural network and the \acrshort{cnn} \& \acrshort{rnn} models.\\
Section~\ref{subsec:compressionTypes} - \textbf{Compression Types}: ...\\
Section~\ref{subsec:AIaccelerators} - \textbf{AI accelerators} stuff about AI accelerators\\
Section~\ref{subsec:hardwareArch} - \textbf{Memory factors for Deep Neural Networks}: brief stuff about this section

\subsection{Deep Neural Networks}\label{subsec:deepLearning}
\subfile{Sections/Background/DeepNeuralNetworks}

\newpage
\subsection{Neural Network Compression}\label{subsec:compressionTypes}
\subfile{Sections/Background/Compression}

\newpage
\subsection{AI accelerators}\label{subsec:AIaccelerators}
\subfile{Sections/Background/AIaccelerators}

\subsection{Memory factors for Deep Neural Networks}\label{subsec:hardwareArch}
\subfile{Sections/Background/HardwareMemArch}

\pagebreak
\section{Research Methodology}
\subfile{Sections/ResearchMethodology}

\pagebreak
\section{Design}
\subfile{Sections/Design}

\pagebreak
\section{Evaluation Strategy}
\subfile{Sections/EvaluationStrat}

\pagebreak
\section{Project Management}
\subfile{Sections/ProjectManagement}

\pagebreak
\appendix
\section{Back matter}
\subsection{References}
\printbibliography

\subsection{Appendices}
to include additional material, consult with your supervisor.

\printnoidxglossary[type=acronym]
\printacronyms

\end{document}
